\documentclass[a4paper]{article}

\usepackage{polski}
\usepackage[utf8]{inputenc}

\usepackage[export]{adjustbox}
\usepackage{scrextend}
\usepackage{amsfonts}
\usepackage{amsmath}
\usepackage{svg}

\usepackage{geometry}
\geometry{a4paper, left=15mm, top=30mm, right=15mm, bottom=20mm}

\usepackage{gensymb}
\usepackage{graphicx} 
\usepackage{isotope}
\usepackage{array}
\usepackage{float}
\usepackage{titlesec}
\usepackage{fancyhdr}
\usepackage{multirow}

\usepackage{hyperref}
\usepackage{sectsty}
\usepackage{enumitem}
\usepackage{listings}
\usepackage[labelformat=simple]{subcaption}
\usepackage{xcolor,colortbl}
\usepackage{animate}

\sectionfont{\normalfont\huge\sectionrule{0pt}{0pt}{-6pt}{1pt}}
\subsectionfont{\normalfont\LARGE}

\pagestyle{fancy}
\fancyhf{}
\fancyhead[LE,LO]{\Large Łukasz Kwinta}
\fancyhead[LE,RO]{\Large Laboratorium 3 - Triangulacja Wielokątów Monotonicznych}
\fancyfoot[CE,CO]{\Large\thepage}

\renewcommand{\footrulewidth}{1pt}
\renewcommand{\headrulewidth}{1pt}

\definecolor{Gray}{gray}{0.85}
\definecolor{LightGray}{gray}{0.95}

\newcolumntype{a}{>{\columncolor{Gray}}c}
\newcolumntype{b}{>{\columncolor{white}}c}

\hypersetup{
    colorlinks,
    citecolor=black,
    filecolor=black,
    linkcolor=black,
    urlcolor=black
}

\title{\fontsize{30pt}{30pt}\selectfont Laboratorium 3 \\ Triangulacja Wielokątów Monotonicznych}
\author{\fontsize{20pt}{20pt}\selectfont Łukasz Kwinta}
\date{}

\begin{document}
\maketitle
\Large
\vspace*{\fill}
\section{Dane Techniczne}
Procesor: AMD Ryzen 7 5700U\\
System operacyjny: Ubuntu 20.04 w środowisku WSL 2 na Windows 11 x64\\
Pamięć ram: 32 GB DDR4\\
\\
\\
Środowisko i język: Python 3.9 + Jupyter Notebook w środowisku Anaconda\\
Wykresy tworzyłem przy pomocy narzędzia przygotowanego przez KN Bit, 
do obliczeń numerycznych używałem biblioteki numpy.
 Dane przechowywałem w zmiennych typu float – typ danych o rozmiarze 64 bitów, 
 odpowiednik typu double w języku C.
\pagebreak
\section{Opis Realizacji Ćwiczenia}
Celem ćwiczenia była implementacja algorytmów Grahama i Jarvisa do wyznaczania
otoczki wypukłej chmury punktów oraz porównanie ich precyzji oraz wydajności.
\end{document}